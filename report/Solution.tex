\documentclass[a4paper,12pt]{article}
\usepackage[english]{babel}
\usepackage[utf8x]{inputenc}
\usepackage{hyperref}
\usepackage{graphicx}
\usepackage{caption}
\usepackage{floatrow}


\title{\scshape Casper wargame \\ Solutions}
\author{Francesca Del Nin \\ student number: 0734630}
\date{December 2018}

\begin{document}
\maketitle

\tableofcontents

\newpage
\section{Overview}
Passwords retrieved:
\begin{table}[]
\begin{tabular}{l|l|l}
 Level & Password & Time spent  \\
 \hline
 casper4 & zssEylQhyfOdX0H7OOKFxEXG0iY9Y7PL &   \\
 casper40 & hcRzYRX24hMUQfGN4comHv8pXgl7rgiK & \\
 casper41 & qQftts3uK2DjrNxfEPzXTyOpB20vMakY & \\
 casper42 & 1RbjDY7vpb1RvjvLiAN9skfLL3MTZABD & \\
  \hline
 casper5 & LzroSg7w54LlHxKZbHVEGYj2UPs1WEUG & \\
 casper50 & SuzPtkbto6T8w0H43MQimqOoH7ntIoVh & \\
 casper51 & TOv1elxDKtaFgGfWTyyLQ2uzivxI36X4 & \\
 casper52 & rWhAgxz2U89eRx57LyCmEWunbCW39fAO & \\
  \hline
 casper6 & l1BcWzPZaWMGOM0kaWimJmmT40KmDE9o &  \\
 casper60 & EoWZv5u7pnBkchLK30vXhiyH1sDexpCB & \\
 casper61 & K14irUGErx7IC3Gvh5ZdO5sIKt5eQCde & \\
 casper62 & qe3YqqaQSkvJhTEMxt68uUwemCc9d1WI & \\
  \hline
 casper8 & AAqtUL09LWefJlJFvTg0SpVg0j89DGe2 & \\
 casper80 & 5CS80xbUrxgjbrU3BBS5YslY00qoUd5J & \\
 casper81 & yMxdglE84E0NpXIXXRcukr8uEWkIHE6w & \\
 casper82 & MABIQpR4omrMCmdc7PwxbRNm5EPHzYX5 & \\
  \hline
 casper10 & BpI02agB8PmOKDI1LRvMeQJWi2GrzgWA &\\
\end{tabular}
\end{table}

\newpage
\section{Casper4 solution}
\subsection{Description}
Program casper4 takes an input from the user and prints a greeting and the input from the user. If no input is provided it prints an explanation on how to use the program and exits. The important variables are:
\begin{itemize}
\item \texttt{char buf[666]} inside greetUser,
\item \texttt{char *s} argument of greetUser function
\end{itemize}
 

\subsection{Vulnerability}

This program is vulnerable because it calls \texttt{strcpy} function passing the (char) pointer \texttt{s} without doing a bound check on the length of the variable pointed, so it can be longer than 666 byte (or the 674 allocated), which is the length of the \texttt{buf} array in which the content will be copied.

\subsection{Exploit description}

To exploit this level I first needed to know how many characters fill the buffer and overwrite the return address of \texttt{greetUser}. To find this number I used gdb, put a breakpoint inside the \texttt{greetUser} function and run the program, the address found is 0xbffff146. Then, by using \texttt{info frame} command I am able to see at which address is stored the eip register where the return address is stored, the address is 0xbffff6bc. The distance between the two is 678 byte, so by adding another 4 byte I'm able to make a buffer overflow attack.

I provide as input a string of 682 byte in total composed as following:
\begin{itemize}
\item 100 NOP,
\item the shellcode (21 byte),
\item some padding to fill the buffer and reach the return address (557 byte),
\item the address of the buffer.
\end{itemize}

%./casper4 $(python -c "print('\x31\xc9\xf7\xe1\x51\x68\x2f\x2f\x78\x68\x68\x2f\x62\x69\x6e\x89\xe3\xb0\x0b\xcd\x80'+'A'*(674-21)+'\x16\xf4\xff\xbf');")

In this way I fill the whole buffer and overwrite the return address (and the frame pointer too) of the function, so when the execution of \texttt{greetUser} is done, the instruction inside the buffer (so the shellcode) will be executed instead of returning to main. 

The NOP instructions at the beginning of the buffer are to make sure that even if the address of the buffer is not precise (due to the program being run inside gdb), the shellcode will be executed because it is preceded just by NOP instructions. To make sure the overwritten return address will point somewhere between the NOP before the shellcode and not to too low I changed a bit the address of the buffer by putting a slightly higher number, in this way even if the addresses are bit different inside and outside gdb the attack works(final address: 0xbffff186).

%To find the distance between the buffer and the return address there are tw

\subsection{Mitigation}

Different things can be done to mitigate this attack: checking the length of the string copied to the buffer is a first thing that can void this attack. It's possible to do this by using the \texttt{strncpy} instead of \texttt{strcpy} which takes three arguments: the destination, the source (what I want to copy) and the maximum number of characters that can be copied.

Another thing that can be done is making the stack non-executable, so even in case of a buffer overflow no code inside the stack will be executed, but the overwrite of the return address is still possible. To make it more hard it's possible to add canaries that make easier to detect this kind of attack. Another measure that can help is enabling the ASLR (Address Space Layout Randomization) that makes it harder to get the correct address of the buffer since it changes at every execution of the program.

\subsection{Advanced level casper41}

This exploit works also with casper41 because it only checks for environment variables which are not being used for the exploit. 

\subsection{Advanced level casper40}

This level has an additional check compared to level4: it looks for NOP instructions inside the string provided in input. To exploit this level I used a different approach: instead of storing the shellcode inside the buffer I stored it in a new environmental variable (together with some NOP instruction).  

In this way I am able to fill the buffer with A's and the address of the variable (which does not contain \textbackslash x90)

Also the distance between the buffer and the return address is changed, now I need 682 byte to reach it (686 to overwrite it).

So the string provided in input is:
\begin{itemize}
\item 682 A's
\item the address of the environmental variable
\end{itemize}

To get the correct address of the environmental variable I used gdb \texttt{x/s *((char **)environ))} which shows the environmental vars and their address. I found out that the address is 0xbffff8b4. To make this exploit work I changed a bit the address to be sure that it points between the NOP's (final address 0xbffff880).

As last thing in the bash script I remove the environmental variable.

\subsection{Advanced level casper42}

This level checks for non-ascii characters inside the buffer but just for its expected legnth that is 666 bytes, it does not check for the entire length of the string provided in input. To exploit this level I used the same approach of level 40 since it fills the buffer with A's characters and only after (more than) 666 byte there is the first non ascii character(the address of the environmental variable).

\section{casper5}

\subsection{Description}
This level is similar to casper4 but it takes the input from stdin instead of as argument from the terminal.
The most important variable is \texttt{char buf[666]} inside greetUser.


\subsection{Vulnerability}
This program is vulnerable because it uses \textbf{gets} function to read the input, this function does not check for the length of the input, so an attacker can use this vulnerability to overflow the buffer and overwrite the return address of \texttt{greetUser}.


\subsection{Exploit description}
Using gdb I discovered how many characters I need to overwrite the return address of \texttt{greetUser} from the buffer, which is 678+4 bytes, and the address of the buffer.

The string that is provided as input is composed as follows:
\begin{enumerate}
\item Some padding characters (100 Nop),
\item The shellcode,
\item some other padding to fill the buffer and reach the return address (557 NOP)
\item the address of the buffer.
\end{enumerate}

Keeping the input open (with cat -) makes it possible to use the shell when it's launched.


\subsection{Mitigation}

To mitigate this attack the length of the input provided by the user should be checked, the function \texttt{gets} should be replaced with the safer version \texttt{fgets} which has an additional arguments that tells how many characters can by copied.

\subsection{Advanced level casper51}
The same exploit of the level casper5 works also for this program because in casper51 is added a control on the environmental variables which are not used for the attack. 


\subsection{Advanced level casper50}

To exploit this level, which checks for NOP instruction for the entire length of the input provided I used an environmental variable to store the shellcode (together with some NOP's), filled the buffer with A's and overwrite the return address with the address of the environmental variable.

In this case I also needed to add four bytes in the input string since the distance between the buffer and the return address has changed.

\subsection{Advanced level casper52}

This level checks non ascii characters for the expected length of the buffer (666 bytes), the same exploit used in casper50 works also for this level because the first 666 characters are A's.

\section{casper6}
% ./casper6 $(python -c "print('\x90'*(668-21)+'\x31\xc9\xf7\xe1\x51\x68\x2f\x2f\x78\x68\x68\x2f\x62\x69\x6e\x89\xe3\xb0\x0b\xcd\x80'+'\x01\x98\x04\x08');")


\subsection{Description}

This program prints a greeting to the user when provided an input, it uses a struct composed by a buffer array and a function pointer. The relevant variables are:
\begin{itemize}

\item \texttt{somedata} struct which contains:
		\begin{itemize}
		\item \texttt{char buf[666]}
		\item \texttt{void (*fp)(char *)} function pointer
		\end{itemize}
\item \texttt{argv[1]} in main which is the input provided by the user
\end{itemize}

\subsection{Vulnerability}

This program is vulnerable because it is possible to overwrite the function pointer of the struct, which is stored just above the buffer. The struct is not stored in the stack because it's a global variable and it's stored in the data segment of the memory, so even if the stack is non-executable it's still possible to run an overflow attack.


\subsection{Exploit description}

To exploit this level I get the address of \texttt{somedata.buffer} and \texttt{somedata.fp} through gdb, in this way I know how many characters I need to reach the function pointer and overwrite its content (668 bytes).

The string I pass as input is composed as follows:

\begin{itemize}
\item some NOP instruction(597 bytes)
\item shellcode (21 bytes)
\item some more NOP to fill the buffer an reach the function pointer (50 bytes)
\item the address of the buffer (0x08049801)
\end{itemize}

Since the address of the buffer running the program inside gdb and without it can vary a bit, some NOP instructions at the beginning are necessary to make sure that the exploit still works.

The address of the buffer appears to contain the \textbackslash x00 value, which is ignored by the shell, so I changed the last byte to 01. In this case this small changes it's enough to land inside the buffer between the NOP instructions, but it's also possible to write a higher address to be sure that it's pointing inside the NOPs even if the address is changed a bit.  

\subsection{Mitigation}

To mitigate this attack a check of the length of the string copied it's necessary, this can be done by replacing \texttt{strcpy} with the safer version \texttt{strncpy} which also has the length parameter.

Another thing that can help mitigate this attack is using ASLR(Address Space Layput Randomization)
which makes more difficult retrieving the address of the buffer since it will vary at every execution of the program.

Also if the buf variable and the function pointer are stored in the revers order this attack it's not possible because overflowing the buffer would not lead to the function pointer.

\subsection{Advanced level casper61}

This program adds an additional check on the environmental variables (that can be used to store the shellcode), which I did not use in exploiting level6, so the same approach can be used. In this case I changed the address of the buffer since it has changed to 0x8049860.

\subsection{Advanced level casper60}

This levels checks for NOP instruction inside the string provided as input. To exploit this level I stored the shellcode inside an environmental variable and filled the buffer with A's and the address of the variable.

Together with the shellcode I stored also some NOP instruction to make it easier to get an address that makes the attack work.

To get the address of the environmental variable I exported the variable and used gdb to get the address, then I changed the address to a slightly higher one to make sure it will point inside the NOP instructions.

The string provided as input is:
\begin{itemize}
\item 668 A's
\item the address of the environmental variable (0xbffff8a0)
\end{itemize}

\subsection{Advanced level casper62}

The same exploit of casper60 works for casper62 because this level checks for non ascii character but only for the expected length of the buffer (666 bytes) and not for the entire string provided.

\section{casper8}
\subsection{Description}

This level prints a greeting to the user as the other levels, but in this case the stack is non-executable so it's not possible to do a stack based buffer overflow, but canaries are not enabled. The code it's the same as casper4.
The relevant variable are:

\begin{itemize}
\item \texttt{char buf[666]} inside greetUser,
\item \texttt{char *s} argument of greetUser function
\end{itemize}

\subsection{Vulnerability}

In this case the vulnerability is given by the use of the \texttt{strcpy} function that allow an attacker to overflow the return address of greetUser, and the loaded libc library allows to call the system function and launch a shell.

\subsection{Exploit description} 

To make this exploit work I first looked for how many input characters make the program go in segmentation fault, then I verified that I was overwriting the whole EIP register (where the return address is stored) using the "info frame" command after the execution of \texttt{strcpy} inside \texttt{greetUser}, in this way I discovered that I need a total of 682 byte to completely overwrite the return address of \texttt{greetUser}.

Then I searched for the address of the \texttt{system} function (which execute a shell command) inside libc, I used gdb command \texttt{p \&system} while running the program using make.

Since I want to pass the string "/bin/xh" that is nowhere inside the library I can put it inside the buffer because even if the stack is non executable I can still use it to store arguments that can be passed to functions. In this case instead of NOP I used spaces to make some padding before and after "/bin/xh".

To make it possible to store spaces I can pass the variable inside the bash script through "". To make it a cleaner attack I also searched for the address of the \texttt{exit} function inside the library to pass it as return address of the system function, in this case the program exits instead of fall into segmentation fault.

The string in input looks as follows:
\begin{enumerate}
\item A string composed by a lot of spaces, "/bin/xh" and some other spaces (total of 678 bytes),
\item \texttt{system} function address,
\item \texttt{exit} function address,
\item the address of the string.
\end{enumerate}

To find the address of the string I used some A's and B's before and after /bin/xh respectively, in this way the first time I run the program (using the buffer address) I was able to see where I was pointing (at some B's) and thus change a bit the address until /bin/xh showed up. At this point I put spaces and the attack worked.




% Since I want to pass as argument to this function the string "/bin/xh", which is not present inside the loaded library, I created a new environment variable, since the exact address can vary a bit the variable contains some spaces and only then "/bin/xh".
 %To find out the address in which is stored I used the command \texttt{x/s *((char **)environ)} and went through the variables until I found MYSHELL at   *((char **)environ+19)% at the address 0xbfffff3f
%.


%This attack worked inside gdb but not outside because the address of the environmental variables can change a bit, the error displayed was telling "bin/xh: not found" so the string passed was missing a "/". Changing the address of the string to 0xbfffff46 did the trick.

\subsection{Mitigation}

Checking for the length of the string in input before copying it into the buffer can mitigate this attack, to do this \texttt{strncpy} function can be used instead of \texttt{strcpy}.

Also enabling ASLR (Address Space Layout Randomization) makes the attack more difficult because the location of the stack, heap and library can vary each time the program is launched, so it becomes difficult to retrieve the address of the functions to which an attacker wants to jump.

\subsection{Advanced level casper81}

This level checks for the environmental variables which I did not use in exploiting level8, so the same exploit can be used.

\subsection{Advanced level casper80}

The same approach work also for this advanced level because it only checks for NOP in the input. To make it work I checked again the distance from the buffer to the return address of \texttt{greetUser} function, this time it has changed to 682 byte (so 686 to overwrite it). Adding four more padding spaces made the attack work.
 % I also changed the address of the string because the address of the environmental variable MYSHELL changes a bit, to make it work I used 0xbfffff44. To get this value I read the error message displayed with the address used in the other exploit and there where two characters missing.



\subsection{Advanced level casper82}

This program checks for non-ascii characters inside the buffer, but just for the expected length (666 byte), so the same attack of the other levels still works.

In particular the attack is the same as casper80 which has 682 byte between the buffer and the return address of the function (686 to overwrite it).


\section{casper10}

\subsection{Description}
This program prints a greeting to the user if it's provided an input string, then checks for the admin flag and if it's true it launches a shell. 

The relevant variables are:
\begin{itemize}

\item \texttt{argv[1]} in main which is the input provided by the user.
\item \texttt{char buf[666]} inside greetUser,
\item \texttt{char *s} argument of greetUser function

\end{itemize}


\subsection{Vulnerability}

This programs contains a format-string vulnerability, this makes it possible to change the value of the \texttt{is Admin} flag even if canaries and non-executable stack are enabled.

A format function like \texttt{printf} expect as input a format string, that is a string that contains text and format specifiers. If the attacker provides in input format parameters (specified by \%) inside a string it can read and write in memory. 

In particular:
\begin{itemize}
\item \texttt{\%n} specifier allow to write the number of characters that have already been formatted to a specific place in memory,
\item and \texttt{\%x} prints the value of an integer, to do this pop values out of the stack.
\end{itemize} 

So by using \texttt{\%x} an attacker can reach a place in memory where he/she wants to write and then use \texttt{\%n} to write in that place in memory.

 %for example the parameter \%x read data from the stack and prints it %pops from stack.

\subsection{Exploit description}

To make this attack work I have to retrieve the address of \texttt{isAdmin} variable and use the format string vulnerability to change it (any value except 0 will work since it's a binary variable). 

To do so I will use the \texttt{\%n} parameter which writes the number of bytes already printed, into a variable we can choose, in this case \texttt{isAdmin}. Before being able to do so I need to move the stack pointer to a location where I can store the address of \texttt{isAdmin}, for example the beginning of the buffer.

To move at the beginning of the buffer I use \texttt{\%x}, which reads from the stack and goes towards lower addresses (so towards the bottom of the stack), the buffer is stored in a lower address because is pushed into the stack before the call to the function \texttt{printf}.

Filling the buffer with some A's characters and some \texttt{\%80x} parameters (08 makes it print in 8 digit exadecimal number) will print back the A's (x41), at this point I know the number of \texttt{\%x} parameter I need to reach the beginning of the buffer.

At this point if I write the address of the variable instead of the first four 'A's and add a \texttt{\%n} parameter I'm able to write some value in that address, or in other words change the value of the flag.

The string provided as input is:
\begin{itemize}
\item the address of the \texttt{isAdmin} variable,
\item nine \texttt{\%80x} parameters to reach the beginning of the buffer,
\item \texttt{\%n} to write in the address.
\end{itemize}


\subsection{Mitigation}


To make this program more secure it's better enable ASLR (Address Space Layout Randomization,)that will make more difficult to retrieve the address of the variable an attacker wants to change. 


\section{Notes}
\begin{enumerate}


\item The shellcode used in all levels that require it is the one suggested in the assignment website, with one byte changed to launch /bin/xh instead of /bin/sh (replacing \textbackslash x73 with \textbackslash x78).


\item All attacks can be launched using \texttt{make exploit\textit{n}}, all attacks except level 8 (and 80 81 82) can also be launched with \texttt{./exploit\textit{n}.sh}
\end{enumerate}

\end{document}
